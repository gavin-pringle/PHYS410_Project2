\documentclass[10pt]{article}

\usepackage[utf8]{inputenc}
\usepackage{latexsym,amsfonts,amssymb,amsthm,amsmath}
\setlength{\parindent}{0in}
\setlength{\parskip}{\baselineskip}
\setlength{\oddsidemargin}{0in}
\setlength{\textwidth}{6.5in}
\setlength{\textheight}{8.8in}
\setlength{\topmargin}{0in}
\setlength{\headheight}{18pt}

\usepackage[a4paper,margin=1in,footskip=0.25in]{geometry}

\usepackage{listings}
\usepackage{color} %red, green, blue, yellow, cyan, magenta, black, white
\definecolor{mygreen}{RGB}{28,172,0} % color values Red, Green, Blue
\definecolor{mylilas}{RGB}{170,55,241}

\usepackage{graphicx}
\graphicspath{{../output/}}

\def\code#1{\texttt{#1}} % Monospacing shortcut: Use \code{}

\usepackage[colorlinks=true, urlcolor=blue, linkcolor=blue]{hyperref}

\usepackage{verbatim} % For including raw text files in the pdf
\usepackage{float} % For keeping figures in the section where they were called 

\title{PHYS 410 Project 2}
\author{Gavin Pringle, 56401938}

%%%%%%%%%%%%%%%%%%%%%%%%%%%%%%%%%%%%%%%%%%%%%%%%%%%%%%%%%%%%%%%%%%%%%%%%%%%%%%%%%%%%%%%%%%%%%%%%%%%%%%%
% Start of document
%%%%%%%%%%%%%%%%%%%%%%%%%%%%%%%%%%%%%%%%%%%%%%%%%%%%%%%%%%%%%%%%%%%%%%%%%%%%%%%%%%%%%%%%%%%%%%%%%%%%%%%
\begin{document}

\maketitle

\lstset{language=Matlab,%
    %basicstyle=\color{red},
    breaklines=true,%
    morekeywords={matlab2tikz},
    keywordstyle=\color{blue},%
    morekeywords=[2]{1}, keywordstyle=[2]{\color{black}},
    identifierstyle=\color{black},%
    stringstyle=\color{mylilas},
    commentstyle=\color{mygreen},%
    showstringspaces=false,%without this there will be a symbol in the places where there is a space
    numbers=left,%
    numberstyle={\tiny \color{black}},% size of the numbers
    numbersep=9pt, % this defines how far the numbers are from the text
    emph=[1]{for,end,break},emphstyle=[1]\color{red}, %some words to emphasise
    %emph=[2]{word1,word2}, emphstyle=[2]{style},
    literate=%
        {ö}{{\"o}}1 % Maps ö to its LaTeX representation
        {Ψ}{{$\Psi$}}1 % Maps Ψ to its LaTeX representation 
        {ψ}{{$\psi$}}1 % Maps ψ to its LaTeX representation       
}

%%%%%%%%%%%%%%%%%%%%%%%%%%%%%%%%%%%%%%%%%%%%%%%%%%%%%%%%%%%%%%%%%%%%%%%%%%%%%%%%%%%%%%%%%%%%%%%%%%%%%%%
% Introduction
%%%%%%%%%%%%%%%%%%%%%%%%%%%%%%%%%%%%%%%%%%%%%%%%%%%%%%%%%%%%%%%%%%%%%%%%%%%%%%%%%%%%%%%%%%%%%%%%%%%%%%%
\subsection*{Introduction}

In this project, the time-dependent Schrödinger equation is solved numerically in both one dimension
and two dimensions. In both cases, solutions account for a time-independent potential term $V$, which
takes the form of a rectangular barrier or well, a double slit (in 2d only), or is zero everywhere. 
The MATLAB script \code{sch\_1d\_cn.m} implements the Crank-Nicolson discretization approach to solve 
the Schrödinger equation in 1d, while the script \code{sch\_2d\_adi.m} implements the 
alternating-direction-implicit (ADI) method to solve the equation in 2d. 

The 1d case is tested by conducting convergence testing in the file \code{ctest\_1d.m} which checks 
for solution convergence among increasing discretization levels. A similar convergence test is done 
for two dimensions in the file \code{ctest\_2d.m}. For the 1d case, the solution "excess fractional
probability" is also examined in the files \code{barrier\_survey.m} and \code{well\_survey.m}, which
provides insights into how much time the quantum particle is spending in a certain location. Lastly,
videos of the 2d wave function scattering off a rectangular barrier or well, and producing 
self-interference through a double slit are created using the scripts \code{video\_rec\_bar.m},
\code{video\_rec\_well.m}, and \code{video\_double\_slit.m}.

%%%%%%%%%%%%%%%%%%%%%%%%%%%%%%%%%%%%%%%%%%%%%%%%%%%%%%%%%%%%%%%%%%%%%%%%%%%%%%%%%%%%%%%%%%%%%%%%%%%%%%%
% Review of Theory
%%%%%%%%%%%%%%%%%%%%%%%%%%%%%%%%%%%%%%%%%%%%%%%%%%%%%%%%%%%%%%%%%%%%%%%%%%%%%%%%%%%%%%%%%%%%%%%%%%%%%%%
\subsection*{Review of Theory}

\subsubsection*{1d Schrödinger Equation}

The 1d Schrödinger Equation PDE is given by the following equation:
% PDE and BCs

% family of exact solutions 

% Include running integral


\subsubsection*{2d Schrödinger Equation}

% PDE and BCs

% family of exact solutions 


%%%%%%%%%%%%%%%%%%%%%%%%%%%%%%%%%%%%%%%%%%%%%%%%%%%%%%%%%%%%%%%%%%%%%%%%%%%%%%%%%%%%%%%%%%%%%%%%%%%%%%%
% Numerical approach
%%%%%%%%%%%%%%%%%%%%%%%%%%%%%%%%%%%%%%%%%%%%%%%%%%%%%%%%%%%%%%%%%%%%%%%%%%%%%%%%%%%%%%%%%%%%%%%%%%%%%%%
\subsection*{Numerical Approach}

\subsubsection*{1d Schrödinger Equation}

% CN equation

% Expanded form for finding tridiagonal coefficients 

% Potentials

\subsubsection*{2d Schrödinger Equation}

% ADI equation
\begin{align}\label{ADI_1}
\left( 1 - i \frac{\Delta t}{2} \partial_{xx}^h \right) \psi_{i,j}^{n+\frac{1}{2}} &=
\left( 1 + i \frac{\Delta t}{2} \partial_{xx}^h \right)
\left( 1 + i \frac{\Delta t}{2} \partial_{yy}^h - i \frac{\Delta t}{2} V_{i,j} \right) \psi_{i,j}^n,
\nonumber \\ i &= 2,3,\ldots,n_x-1, \quad j = 2,3,\ldots,n_y-1, \quad n = 1,2,\ldots,n_t-1.
\end{align}
\begin{align}\label{ADI_2}
\left( 1 - i \frac{\Delta t}{2} \partial_{yy}^h + i \frac{\Delta t}{2} V_{i,j} \right) 
\psi_{i,j}^{n+1} &= \psi_{i,j}^{n+\frac{1}{2}},
\nonumber \\ i = 2,3,\ldots,n_x-1, \quad j &= 2,3,\ldots,n_y-1, \quad n = 1,2,\ldots,n_t-1.
\end{align}

% Expanded LHS form for finding tridiagonal coefficients 

% Potentials


%%%%%%%%%%%%%%%%%%%%%%%%%%%%%%%%%%%%%%%%%%%%%%%%%%%%%%%%%%%%%%%%%%%%%%%%%%%%%%%%%%%%%%%%%%%%%%%%%%%%%%%
% Implementation
%%%%%%%%%%%%%%%%%%%%%%%%%%%%%%%%%%%%%%%%%%%%%%%%%%%%%%%%%%%%%%%%%%%%%%%%%%%%%%%%%%%%%%%%%%%%%%%%%%%%%%%
\subsection*{Implementation}

\subsubsection*{1d Tridiagonal System}

% Reference sch_1d_cn  

\subsubsection*{2d Tridiagonal System}

% Reference sch_2d_adi

%%%%%%%%%%%%%%%%%%%%%%%%%%%%%%%%%%%%%%%%%%%%%%%%%%%%%%%%%%%%%%%%%%%%%%%%%%%%%%%%%%%%%%%%%%%%%%%%%%%%%%%
% Results
%%%%%%%%%%%%%%%%%%%%%%%%%%%%%%%%%%%%%%%%%%%%%%%%%%%%%%%%%%%%%%%%%%%%%%%%%%%%%%%%%%%%%%%%%%%%%%%%%%%%%%%
\subsection*{Results}

\subsubsection*{1d Schrödinger Equation}

% Describe convergence test (don't need to explain from scratch) and show results

% Describe surveys (explain what lnFe is) and show results 

\subsubsection*{2d Schrödinger Equation}

% Describe convergence test (don't need to explain from scratch) and show results

% Describe numerical experiments, show a screenshot from each (double slit show interference)


%%%%%%%%%%%%%%%%%%%%%%%%%%%%%%%%%%%%%%%%%%%%%%%%%%%%%%%%%%%%%%%%%%%%%%%%%%%%%%%%%%%%%%%%%%%%%%%%%%%%%%%
% Conclusions
%%%%%%%%%%%%%%%%%%%%%%%%%%%%%%%%%%%%%%%%%%%%%%%%%%%%%%%%%%%%%%%%%%%%%%%%%%%%%%%%%%%%%%%%%%%%%%%%%%%%%%%
\subsection*{Conclusions}

% Summarize work 
% Discuss how mostly successful except for perhaps well survey.

% Talk about performance issues. Could fix by preallocatting memory and using parallel computation.

Generative AI was used to help with understanding how to use MATLAB's \code{contourf} function for 
making videos of numerical experiments in 2d. It was also used for help with typesetting this document.

\pagebreak

%%%%%%%%%%%%%%%%%%%%%%%%%%%%%%%%%%%%%%%%%%%%%%%%%%%%%%%%%%%%%%%%%%%%%%%%%%%%%%%%%%%%%%%%%%%%%%%%%%%%%%%
% Appendices
%%%%%%%%%%%%%%%%%%%%%%%%%%%%%%%%%%%%%%%%%%%%%%%%%%%%%%%%%%%%%%%%%%%%%%%%%%%%%%%%%%%%%%%%%%%%%%%%%%%%%%%

\subsection*{Appendix A - sch\_1d\_cn.m Code}
\lstinputlisting{../src/problem1/sch_1d_cn.m}
\pagebreak

\subsection*{Appendix B - ctest\_1d.m Code}
\lstinputlisting{../src/problem1/ctest_1d.m}
\pagebreak

\subsection*{Appendix C - barrier\_survey.m Code}
\lstinputlisting{../src/problem1/barrier_survey.m}
\pagebreak

\subsection*{Appendix D - well\_survey.m Code}
\lstinputlisting{../src/problem1/well_survey.m}
\pagebreak

\subsection*{Appendix E - sch\_2d\_adi.m Code}
\lstinputlisting{../src/problem2/sch_2d_adi.m}
\pagebreak

\subsection*{Appendix F - ctest\_2d.m Code}
\lstinputlisting{../src/problem2/ctest_2d.m}
\pagebreak

\subsection*{Appendix G - video\_rec\_bar.m Code}
\lstinputlisting{../src/problem2/video_rec_bar.m}
\pagebreak

\subsection*{Appendix H - video\_rec\_well.m Code}
\lstinputlisting{../src/problem2/video_rec_well.m}
\pagebreak

\subsection*{Appendix I - video\_double\_slit.m Code}
\lstinputlisting{../src/problem2/video_double_slit.m}
\pagebreak

\end{document}