\documentclass[10pt]{article}

\usepackage[utf8]{inputenc}
\usepackage{latexsym,amsfonts,amssymb,amsthm,amsmath}
\setlength{\parindent}{0in}
\setlength{\parskip}{\baselineskip}
\setlength{\oddsidemargin}{0in}
\setlength{\textwidth}{6.5in}
\setlength{\textheight}{8.8in}
\setlength{\topmargin}{0in}
\setlength{\headheight}{18pt}

\usepackage[a4paper,margin=1in,footskip=0.25in]{geometry}

\usepackage{listings}
\usepackage{color} %red, green, blue, yellow, cyan, magenta, black, white
\definecolor{mygreen}{RGB}{28,172,0} % color values Red, Green, Blue
\definecolor{mylilas}{RGB}{170,55,241}

\usepackage{graphicx}
\graphicspath{{../output/}}

\def\code#1{\texttt{#1}} % Monospacing shortcut: Use \code{}

\usepackage[colorlinks=true, urlcolor=blue, linkcolor=black]{hyperref}

\usepackage{verbatim} % For including raw text files in the pdf
\usepackage{float} % For keeping figures in the section where they were called 

\title{PHYS 410 Project 2}
\author{Gavin Pringle, 56401938}

%%%%%%%%%%%%%%%%%%%%%%%%%%%%%%%%%%%%%%%%%%%%%%%%%%%%%%%%%%%%%%%%%%%%%%%%%%%%%%%%%%%%%%%%%%%%%%%%%%%%%%%
% Start of document
%%%%%%%%%%%%%%%%%%%%%%%%%%%%%%%%%%%%%%%%%%%%%%%%%%%%%%%%%%%%%%%%%%%%%%%%%%%%%%%%%%%%%%%%%%%%%%%%%%%%%%%
\begin{document}

\maketitle

\lstset{language=Matlab,%
    %basicstyle=\color{red},
    breaklines=true,%
    morekeywords={matlab2tikz},
    keywordstyle=\color{blue},%
    morekeywords=[2]{1}, keywordstyle=[2]{\color{black}},
    identifierstyle=\color{black},%
    stringstyle=\color{mylilas},
    commentstyle=\color{mygreen},%
    showstringspaces=false,%without this there will be a symbol in the places where there is a space
    numbers=left,%
    numberstyle={\tiny \color{black}},% size of the numbers
    numbersep=9pt, % this defines how far the numbers are from the text
    emph=[1]{for,end,break},emphstyle=[1]\color{red}, %some words to emphasise
    %emph=[2]{word1,word2}, emphstyle=[2]{style},
    literate=%
        {ö}{{\"o}}1 % Maps ö to its LaTeX representation
        {Ψ}{{$\Psi$}}1 % Maps Ψ to its LaTeX representation 
        {ψ}{{$\psi$}}1 % Maps ψ to its LaTeX representation       
}

%%%%%%%%%%%%%%%%%%%%%%%%%%%%%%%%%%%%%%%%%%%%%%%%%%%%%%%%%%%%%%%%%%%%%%%%%%%%%%%%%%%%%%%%%%%%%%%%%%%%%%%
% Introduction
%%%%%%%%%%%%%%%%%%%%%%%%%%%%%%%%%%%%%%%%%%%%%%%%%%%%%%%%%%%%%%%%%%%%%%%%%%%%%%%%%%%%%%%%%%%%%%%%%%%%%%%
\subsection*{Introduction}

In this project, the time-dependent Schrödinger equation is solved numerically in both one dimension
and two dimensions. In both cases, solutions account for a time-independent potential term $V$, which
takes the form of a rectangular barrier or well, a double slit (in 2d only), or is zero everywhere. 
The MATLAB script \code{sch\_1d\_cn.m} implements the Crank-Nicolson discretization approach to solve 
the Schrödinger equation in 1d, while the script \code{sch\_2d\_adi.m} implements the 
alternating-direction-implicit (ADI) method to solve the equation in 2d. 

The 1d case is tested by conducting convergence testing in the file \code{ctest\_1d.m} which checks 
for solution convergence among increasing discretization levels. A similar convergence test is done 
for two dimensions in the file \code{ctest\_2d.m}. For the 1d case, the solution "excess fractional
probability" is also examined in the files \code{barrier\_survey.m} and \code{well\_survey.m}, which
provides insights into how much time the quantum particle is spending in a certain location. Lastly,
videos of the 2d wave function scattering off a rectangular barrier or well, and producing 
self-interference through a double slit are created using the scripts \code{video\_rec\_bar.m},
\code{video\_rec\_well.m}, and \code{video\_double\_slit.m}.

%%%%%%%%%%%%%%%%%%%%%%%%%%%%%%%%%%%%%%%%%%%%%%%%%%%%%%%%%%%%%%%%%%%%%%%%%%%%%%%%%%%%%%%%%%%%%%%%%%%%%%%
% Review of Theory
%%%%%%%%%%%%%%%%%%%%%%%%%%%%%%%%%%%%%%%%%%%%%%%%%%%%%%%%%%%%%%%%%%%%%%%%%%%%%%%%%%%%%%%%%%%%%%%%%%%%%%%
\subsection*{Review of Theory}

\subsubsection*{1d Schrödinger Equation}

The 1d Schrödinger Equation PDE is given by the following equation:
\begin{equation}\label{sch_1d}
i\psi(x,t)_t = -\psi_{xx} + V(x,t)\psi
\end{equation}
where the wave function, $\psi(x,t)$, is complex. The equation is to be solved on the domain
$$0\leq x \leq 1, \quad 0 \leq t \leq t_{\textrm{max}}$$
subject to initial and boundary conditions
\begin{equation}\label{sch_1d_IC}
\psi(x,0) = \psi_0(x)
\end{equation}
\begin{equation}\label{sch_1d_BC}
\psi(0,t) = \psi(1,t) = 0
\end{equation}

The family of exact solutions to (\ref{sch_1d}) is
\begin{equation}\label{1d_exact_soln}
\psi(x,t) = e^{-i m^2 \pi^2 t} \sin(m \pi x)
\end{equation}
where m is a positive integer. 

Since the modulus squared of the wave function represents the probability density, 
$\rho = |\psi|^2 = \psi \psi^*$, the "running integral" of the probability density represents
the probability that the particle is to the left of $x$ at any given time $t$:
\begin{equation}\label{prob_den}
P(x,t) = \int_0^x \psi(\tilde{x}, t) \psi^*(\tilde{x}, t) d\tilde{x}
\end{equation}
Note that equation (\ref{prob_den}) only computes the correct probability if the wave function 
is properly normalized such that $P(1,t) = 1$. Even if it is not so normalized, we should have
$$P(1,t) = \textrm{conserved to level of solution error}$$

\subsubsection*{2d Schrödinger Equation}

The 2d Schrödinger Equation PDE is given by the following equation:
\begin{equation}\label{sch_2d}
\psi_t = i(\psi_{xx} + \psi_{yy}) - iV(x,y)\psi
\end{equation}
where the wave function, $\psi(x,y,t)$, is complex. The equation is to be solved on the domain
$$0\leq x \leq 1, \quad 0 \leq y \leq 1, \quad 0\leq t \leq t_{\textrm{max}}$$
subject to initial and boundary conditions
\begin{equation}\label{sch_2d_IC}
\psi(x,y,0) = \psi_0(x,y)
\end{equation}
\begin{equation}\label{sch_2_BC}
\psi(0,y,t) = \psi(1,y,t) = \psi(x,0,t) = \psi(x,1,t) = 0
\end{equation}

A family of exact solutions to (\ref{sch_2d}) is given by
\begin{equation}\label{2d_exact_soln}
\psi(x,y,t) = e^{-i (m_x^2 + m_y^2) \pi^2 t} \sin(m_x \pi x) \sin(m_y \pi y)
\end{equation}

%%%%%%%%%%%%%%%%%%%%%%%%%%%%%%%%%%%%%%%%%%%%%%%%%%%%%%%%%%%%%%%%%%%%%%%%%%%%%%%%%%%%%%%%%%%%%%%%%%%%%%%
% Numerical approach
%%%%%%%%%%%%%%%%%%%%%%%%%%%%%%%%%%%%%%%%%%%%%%%%%%%%%%%%%%%%%%%%%%%%%%%%%%%%%%%%%%%%%%%%%%%%%%%%%%%%%%%
\subsection*{Numerical Approach}

\subsubsection*{1d Schrödinger Equation}

We discretize the domain by introducing the discretization level $l$, and the ratio of temporal to 
mesh spacings $\lambda$. The script \code{sch\_1d\_cn.m} takes $t_{\textrm{max}}$, $l$, and $\lambda$ 
as parameters.
$$\lambda = \frac{\Delta t}{\Delta x}$$
$$n_x = 2^l + 1$$
$$\Delta x = 2^{-l}$$
$$\Delta t = \lambda \Delta x$$
$$n_t = \textrm{round} (t_{\textrm{max}} / \Delta t) + 1$$

When the Crank-Nicolson discretization approach is applied to equations (\ref{sch_1d}) - 
(\ref{sch_1d_BC}), the following relation is reached:  
\begin{align}\label{CN}
i \frac{\psi_j^{n+1} - \psi_j^n}{\Delta t} = \frac{1}{2} \left( 
\frac{\psi_{j+1}^{n+1} - 2\psi_j^{n+1} + \psi_{j-1}^{n+1}}{\Delta x^2} 
+ \frac{\psi_{j+1}^n - 2\psi_j^n + \psi_{j-1}^n}{\Delta x^2} \right)
&+ \frac{1}{2} V_j^{n+\frac{1}{2}} \left(\psi_j^{n+1} + \psi_j^n \right) \nonumber \\
j = 2, 3, \dots, n_x - 1, \, n = 1, 2, \dots, n_t &- 1
\end{align}
which is subject to the initial and boundary conditions:
$$\psi_1^{n+1} = \psi_{n_x}^{n+1} = 0, \quad n = 1, 2, \dots, n_t - 1$$
$$\psi_j^1 = \psi_0(x_j), \quad j = 1, 2, \dots, n_x$$

If (\ref{CN}) is rearranged such that each term contains a singular $\psi$ term, we reach equation
(\ref{CN_rearranged}). This is the equation that defines the tridiagonal system used for solving the 
1d Schrödinger equation. The method for implementing this equation in code is described later in the 
implementation section.
\begin{align}\label{CN_rearranged}
\frac{1}{2 \Delta x^2}\psi_{j+1}^{n+1} + & \left( \frac{i}{\Delta t} - \frac{1}{\Delta x^2}
-\frac{1}{2}V_j \right) \psi_j^{n+1} + \frac{1}{2 \Delta x^2} \psi_{j-1}^{n+1} \nonumber \\
& = -\frac{1}{2 \Delta x^2} \psi_{j+1}^n + \left(\frac{i}{\Delta t} + \frac{1}{\Delta x^2}
+ \frac{1}{2}V_j \right) \psi_j^n - \frac{1}{2 \Delta x^2} \psi_{j-1}^n
\end{align}

The script \code{sch\_1d\_cn.m} additionally takes parameters \code{idtype}, \code{idpar}, 
\code{vtype}, and \code{vpar}. \code{idtype} and \code{idpar} define the initial conditions, 
which takes the form of an exact family for \code{idtype == 0}:
$$\psi(x,0) = \sin(m \pi x)$$
or of a boosted Gaussian for \code{idtype == 1}: 
$$\psi(x,0) = e^{i p x} e^{-((x - x_0)/\delta)^2}$$
\code{vtype} and \code{vpar} define the time-independent potential which takes the form of no 
potential for \code{vtype == 0}:
$$V(x) = 0$$
or of a rectangular barrier or well for \code{vtype == 1}:
$$V(x) =
\begin{cases} 
0 & \text{for } x < x_{\text{min}}, \\
V_c & \text{for } x_{\text{min}} \leq x \leq x_{\text{max}}, \\
0 & \text{for } x > x_{\text{max}}.
\end{cases} $$

\subsubsection*{2d Schrödinger Equation}

For the 2d case, the domain is discretized in a similar way by introducing the discretization 
level $l$ and the ratio of temporal to mesh spacings $\lambda$. In the 2d scenario, we always 
assume $x$ and $y$ steps are equal. The script \code{sch\_2d\_adi.m} takes $t_{\textrm{max}}$, $l$, 
and $\lambda$ as parameters.
$$\lambda = \frac{\Delta t}{\Delta x} = \frac{\Delta t}{\Delta y}$$
$$n_x = n_y = 2^l + 1$$
$$\Delta x = \Delta y = 2^{-l}$$
$$\Delta t = \lambda \Delta x$$
$$n_t = \textrm{round}(t_{\textrm{max}} / \Delta t) + 1$$

When ADI discretization method is applied to equation (\ref{sch_2d}), the following relations are 
reached:
\begin{align}\label{ADI_1}
\left( 1 - i \frac{\Delta t}{2} \partial_{xx}^h \right) \psi_{i,j}^{n+\frac{1}{2}} &=
\left( 1 + i \frac{\Delta t}{2} \partial_{xx}^h \right)
\left( 1 + i \frac{\Delta t}{2} \partial_{yy}^h - i \frac{\Delta t}{2} V_{i,j} \right) \psi_{i,j}^n,
\nonumber \\ i &= 2,3,\ldots,n_x-1, \quad j = 2,3,\ldots,n_y-1, \quad n = 1,2,\ldots,n_t-1.
\end{align}
\begin{align}\label{ADI_2}
\left( 1 - i \frac{\Delta t}{2} \partial_{yy}^h + i \frac{\Delta t}{2} V_{i,j} \right) 
\psi_{i,j}^{n+1} &= \psi_{i,j}^{n+\frac{1}{2}},
\nonumber \\ i = 2,3,\ldots,n_x-1, \quad j &= 2,3,\ldots,n_y-1, \quad n = 1,2,\ldots,n_t-1.
\end{align}
Where it is understood that $\psi^{n+1}$ is found by first solving for $\psi^{n+\frac{1}{2}}$ via 
equation (\ref{ADI_1}). Both (\ref{ADI_1}) and (\ref{ADI_2}) are subject to the initial and 
boundary conditions:
$$\psi_{i,j}^1 = \psi_0 (x_i, y_j)$$
$$\psi_{1,j}^n = \psi_{n_x,j}^n = \psi_{i,1}^n = \psi_{i,n_y}^n = 0$$
The operators $\partial_{xx}^h$ and $\partial_{yy}^h$ are defined as:
$$\partial_{xx}^h u_{i,j}^n \equiv \frac{u_{i+1,j}^n - 2u_{i,j}^n + u_{i-1,j}^n}{\Delta x^2}$$
$$\partial_{yy}^h u_{i,j}^n \equiv \frac{u_{i,j+1}^n - 2u_{i,j}^n + u_{i,j-1}^n}{\Delta y^2}$$

The left-hand-side of equation (\ref{ADI_1}) can be rearranged into the following form for 
computing the sparse matrix for the first tridiagonal system:
\begin{equation}\label{n_half_LHS}
-\frac{i \Delta t}{2 \Delta x^2} \psi_{i+1,j}^{n+\frac{1}{2}} + \left(1 + 
\frac{i \Delta t}{\Delta x^2} \right) \psi_{i,j}^{n+\frac{1}{2}} - \frac{i \Delta t}{2 \Delta x^2} 
\psi_{i-1,j}^{n+\frac{1}{2}}
\end{equation}
The right-hand-side of equation (\ref{ADI_1}) is not expanded due to its complexity, but is instead
reduced to two steps of computation. Equation (\ref{n_half_RHS_1}) shows the action of the rightmost
parenthesis on $\psi_{i,j}^n$ in equation (\ref{ADI_1}). The output of this action is labelled as 
$f^*$.
\begin{equation}\label{n_half_RHS_1}
\frac{i \Delta t}{2 \Delta y^2} \psi_{i,j+1}^{n} + \left(1 - i \Delta t \left(\frac{1}{\Delta y^2}
+ \frac{V_{i,j}}{2} \right) \right) \psi_{i,j}^n + \frac{i \Delta t}{2 \Delta y^2} \psi_{i,j-1}^n
= f^*
\end{equation}
The action of the second from right parenthesis in equation (\ref{ADI_1}) on $f^*$ is then
shown in equation (\ref{n_half_RHS_2}).
\begin{equation}\label{n_half_RHS_2}
\frac{i \Delta t}{2 \Delta x^2} f_{i+1}^* + \left(1 - \frac{i \Delta t}{\Delta x^2} \right) f_i^*
+ \frac{i \Delta t}{2 \Delta x^2} f_{i-1}^* = f
\end{equation}
Equations (\ref{n_half_LHS}), (\ref{n_half_RHS_1}) and (\ref{n_half_RHS_2}) define the first
tridiagonal system in the ADI scheme, for which the implementation in code is described later in the 
implementation section. 

The output of the above tridiagonal system is then fed into the second tridiagonal system in the ADI 
scheme, which comes from rearranging equation (\ref{ADI_2}):
\begin{equation}\label{n_full}
-\frac{i \Delta t}{2 \Delta y^2} \psi_{i,j+1}^{n+1} + \left(1 + \frac{i \Delta t}{\Delta y^2} + 
\frac{i \Delta t}{2} V_{i,j} \right) \psi_{i,j}^{n+1} - \frac{i \Delta t}{2 \Delta y^2} 
\psi_{i,j-1}^{n+1} = \psi_{i,j-1}^{n+\frac{1}{2}}
\end{equation}
How equations (\ref{n_half_LHS}), (\ref{n_half_RHS_1}), (\ref{n_half_RHS_2}), and (\ref{n_full})
are implemented in code is described in detail in the implementation section of this report. 

The script \code{sch\_2d\_adi.m} additionally takes parameters \code{idtype}, \code{idpar}, 
\code{vtype}, and \code{vpar}. \code{idtype} and \code{idpar} define the initial conditions, 
which takes the form of an exact family for \code{idtype == 0}:
$$\psi(x,y,0) = \sin(m_x \pi x) \sin(m_y \pi y)$$
or of a boosted Gaussian for \code{idtype == 1}: 
$$\psi(x,y,0) = e^{i p_x x} e^{i p_y y} e^{-((x - x_0)^2 / \delta_x^2 + (y - y_0)^2 / \delta_y^2)}$$
\code{vtype} and \code{vpar} define the time-independent potential which takes the form of no 
potential for \code{vtype == 0}:
$$V(x,y) = 0$$
of a rectangular barrier or well for \code{vtype == 1}:
$$V(x, y) = 
\begin{cases} 
V_c & \text{for } (x_{\min} \leq x \leq x_{\max}) \text{ and } (y_{\min} \leq y \leq y_{\max}) \\
0 & \text{otherwise}
\end{cases}$$
or of a double slit potential for \code{vtype == 2}:
$$j' = (n_y - 1) / 4 + 1$$
$$V_{i,j'} = V_{i,j'+1} = 0 \quad \text{for} \quad 
\left[ (x_1 \leq x_i) \ \text{and} \ (x_i \leq x_2) \right] \ \text{or} \ 
\left[ (x_3 \leq x_i) \ \text{and} \ (x_i \leq x_4) \right]$$
$$V_{i,j'} = V_{i,j'+1} = V_c \quad \text{otherwise}$$
$$V_{i,j} = 0 \quad \text{for} \quad j \neq (j' \ \text{or} \ j'+1)$$

%%%%%%%%%%%%%%%%%%%%%%%%%%%%%%%%%%%%%%%%%%%%%%%%%%%%%%%%%%%%%%%%%%%%%%%%%%%%%%%%%%%%%%%%%%%%%%%%%%%%%%%
% Implementation
%%%%%%%%%%%%%%%%%%%%%%%%%%%%%%%%%%%%%%%%%%%%%%%%%%%%%%%%%%%%%%%%%%%%%%%%%%%%%%%%%%%%%%%%%%%%%%%%%%%%%%%
\subsection*{Implementation}

\subsubsection*{1d Tridiagonal System}

For the 1d case, the tridiagonal system is set up outside of the main loop that iterates once per time
step following the left-hand-side of equation (\ref{CN_rearranged}). Boundary conditions are also 
enforced by setting the coefficients in the first and last rows of the tridiagonal matrix. The sparse
matrix is created by using MATLAB's \code{spdiags} function:
\begin{verbatim}
% Set up tridiagonal system
dl = 0.5/dx^2 * ones(nx, 1);
d  = (1i/dt - 1/dx^2 - 0.5*v.') .* ones(nx,1);
du = dl;
% Fix up boundary cases
d(1) = 1.0;
du(2) = 0.0;
dl(nx-1) = 0.0;
d(nx) = 1.0;
% Define sparse matrix
A = spdiags([dl d du], -1:1, nx, nx);
\end{verbatim}

The right-hand-side of equation (\ref{CN_rearranged}) defines the right-hand-side of the linear 
system that is solved each time step, which is given the variable name \code{f}. The system is solved
by using left-division. Boundary conditions are also enforced each time step by setting the first and
last spatial values of $\psi$ to zero.
\begin{verbatim}
% Compute solution using CN scheme
for n = 1 : nt-1
    % Define RHS of linear system
    f(2:nx-1) = psi(n, 2:nx-1) .* (1i/dt + 1/dx^2 + 0.5*v(2:nx-1)) ...
                + (-0.5/dx^2) * (psi(n, 1:nx-2) + psi(n, 3:nx));
    f(1) = 0.0;
    f(nx) = 0.0;
    % Solve system, thus updating approximation to next time step
    psi(n+1, :) = A \ f;
    % Set first and last values to zero
    psi(n+1, 1) = 0; 
    psi(n+1, nx) = 0; 
\end{verbatim}

Refer to Appendix A for the full implementation of \code{sch\_1d\_cn.m}.

\subsubsection*{2d Tridiagonal System}

For the 2d case, the first tridiagonal system is set up outside of the main loop that iterates once 
per time step following equation (\ref{n_half_LHS}). The setup is similar to the 1d case:
\begin{verbatim}
% Define sparse matrix diagonals for first ADI eqn
dl = (-1i*dt/(2*dx^2)) * ones(nx, 1);
d  = (1 + 1i*dt/(dx^2)) * ones(nx, 1);
du = dl;
% Impose boundary conditions
d(1)     = 1.0;
du(2)    = 0.0;
dl(nx-1) = 0.0;
d(nx)    = 1.0;
% Compute sparse matrix for first ADI eqn
A_half = spdiags([dl d du], -1:1, nx, nx);
\end{verbatim}

Two consecutive nested loops are within the main loop that iterates once per time step. The first 
of these loops solves the $n+\frac{1}{2}$ tridiagonal system for each column. Inside the loop, 
the equations (\ref{n_half_RHS_1}) and (\ref{n_half_RHS_2}) are applied in stages: 
\begin{verbatim}
% Solve tridiagonal system for each j (column)
for j = 2:ny-1
    % Array for holding the RHS of the first ADI eqn
    f = zeros(nx,1);

    % Compute RHS of first ADI eqn in stages. 
    f(2:nx-1) = (1i*dt/(2*dy^2)) * (psi_n(2:nx-1, j+1) + psi_n(2:nx-1, j-1)) ...
                + (1 - 1i*dt*(1/dy^2 + v(2:nx-1,j)/2)) .* psi_n(2:nx-1, j);
    f(2:nx-1) = (1i*dt/(2*dx^2)) * (f(1:nx-2) + f(3:nx)) + ...
                (1 - 1i*dt/dx^2) * f(2:nx-1);
\end{verbatim}

The second of these loops solves the $n+1$ tridiagonal system for each row, corresponding to  
equation (\ref{n_full}). The upper and lower diagonals are defined outside of this nested loop as 
they do not depend on the row being solved for, but the middle sparse matrix diagonal is defined
each nested loop iteration as the potential changes as \code{i} is iterated through. The last line
of the code snippet below computes the complete solution for the next time step.
\begin{verbatim}
% Define upper and lower sparse matrix diagonals for second ADI eqn
dl = (-1i*dt/(2*dy^2)) * ones(ny, 1);
du = dl;
% Impose boundary conditions
du(2)    = 0.0;
dl(ny-1) = 0.0;

% Solve tridiagonal system for each i (row)
for i = 2:nx-1
    % Define middle sparse matrix diagonal for second ADI eqn
    v_i = reshape(v(i,:), ny, 1);
    d = 1 + 1i*dt/dy^2 + (1i*dt/2)*v_i;
    % Impose boundary conditions 
    d(1)  = 1.0;
    d(ny) = 1.0;

    % Compute sparse matrix for second ADI eqn
    A_full = spdiags([dl d du], -1:1, ny, ny);

    % Compute RHS of second ADI eqn. BCs already imposed previously
    f = reshape(psi_half(i,:), ny, 1);

    % Solve second ADI system
    psi(n+1, i, :) = A_full \ f;
\end{verbatim}

These two loops form the complete ADI method for solving the 2d Schrödinger equation PDE.
Refer to Appendix E for the full implementation of \code{sch\_2d\_adi.m}.

%%%%%%%%%%%%%%%%%%%%%%%%%%%%%%%%%%%%%%%%%%%%%%%%%%%%%%%%%%%%%%%%%%%%%%%%%%%%%%%%%%%%%%%%%%%%%%%%%%%%%%%
% Results
%%%%%%%%%%%%%%%%%%%%%%%%%%%%%%%%%%%%%%%%%%%%%%%%%%%%%%%%%%%%%%%%%%%%%%%%%%%%%%%%%%%%%%%%%%%%%%%%%%%%%%%
\subsection*{Results}

\subsubsection*{1d Schrödinger Equation}

In order to test the convergence of the 1d Schrödinger Equation, the scaled differences between 
discretization level solutions as well as the scaled exact errors of the exact family solutions at 
multiple discretization levels were computed. The $l$-2 norms (spatial RMS values) of these values 
were then plotted as a function of time to check for convergence. In Figures 1-3 below, we see close 
convergence of curves indicating the solutions are accurate to $O(h^2)$. We will also note that error
appears to accumulate linearly for all plots.

\begin{figure}[H]
\centering
\includegraphics[width=0.95\textwidth]{problem1/ctest_1d-1.png}
\caption{Output 1 of \code{ctest\_1d.m} - 
l-2 norm of difference between discretization level solutions for exact family initial conditions.}
\end{figure}
\begin{figure}[H]
\centering
\includegraphics[width=0.95\textwidth]{problem1/ctest_1d-2.png}
\caption{Output 2 of \code{ctest\_1d.m} - 
l-2 norm of exact error at each discretization level for exact family initial conditions.}
\end{figure}
\begin{figure}[H]
\centering
\includegraphics[width=0.95\textwidth]{problem1/ctest_1d-3.png}
\caption{Output 3 of \code{ctest\_1d.m} - 
l-2 norm of difference between discretization level solutions for boosted Gaussian initial 
conditions.}
\end{figure}

As described earlier, the running integral of the probability density can be taken to determine the 
probability that the quantum particle is to the left of any spatial value $x$
(equation (\ref{prob_den})), assuming proper normalization: $P(x) \rightarrow \bar{P}(x)$. 
Given two values of $x$, we can then compute the probability that the particle lies \textit{between}
$x_1$ and $x_2$. For a free particle, this probability is equal to the long-run fraction of time the 
particle spends between $x_1$ and $x_2$. The \textit{excess fractional probability} that the particle
spends in a given spatial interval is defined as:
\begin{equation}\label{Fe}
\bar{F_e}(x_1, x_2) = \frac{\bar{P}(x_2) - \bar{P}(x_2)}{x_2 - x_1}
\end{equation}
This value is greater than 1 if the particle spends more time between $x_1$ and $x_2$ than a free 
particle would and is less than 1 if it spends less time than a free particle would. The MATLAB 
scripts \code{barrier\_survey.m} and \code{well\_survey.m} were written that simulate the 1d 
wave function with boosted Gaussian initial conditions interacting with a potential barrier and well,
respectively. In Figures 4 and 5, the natural logarithm of $\bar{F_e}$ is plotted vs the natural 
logarithm of the potential of the barrier or well. 

In Figure 4, we can see that the probability that the particle is to the left of the potential 
barrier on $0.6 \leq x \leq 0.8$ decreases dramatically as the height of the potential barrier 
increases. In Figure 5, we can see that the probability that the particle is in the potential well
on $0.6 \leq x \leq 0.8$ oscillates as the depth of the potential well increases, with a general 
trend towards spending \textit{less} time in the potential well as the depth of the well increases.

\begin{figure}[H]
\centering
\includegraphics[width=0.95\textwidth]{problem1/barrier_1d.png}
\caption{Output of \code{barrier\_1d.m} - 
Log-log plot of excess fractional probability vs. $V_0$ for 1d potential barrier.}
\end{figure}
\begin{figure}[H]
\centering
\includegraphics[width=0.95\textwidth]{problem1/well_1d.png}
\caption{Output of \code{well\_1d.m} - 
Log-log plot of excess fractional probability vs. $V_0$ for 1d potential well.}
\end{figure}

\subsubsection*{2d Schrödinger Equation}

As for the 1d case, convergence testing was done for the computed solutions to the 2d Schrödinger 
equation, where the $l$-2 norm was taken across all spatial coordinates, $x$ and $y$. The results 
are shown in Figures 6 and 7. In Figure 6, we see that there is convergence among the $l$-2 norm of
the scaled differences between solutions at different discretization levels, indicating $O(h^2)$
accuracy. However, in Figure 7, we do not see convergence for the $l$-2 norm of the exact solution
errors, and in fact we see quite large errors that increase proportionally to the scaling factor. 
This indicates that either the computed ADI solution, the computed exact solution, or both are 
incorrect by some amount. Thorough debugging could not reveal what is causing this issue, although 
it seems more likely to be caused by the way the exact solution is computed since we see convergence 
for the solutions computed via \code{sch\_2d\_adi.m}.

\begin{figure}[H]
\centering
\includegraphics[width=0.95\textwidth]{problem2/ctest_2d-1.png}
\caption{Output 1 of \code{ctest\_2d.m} - 
l-2 norm of difference between discretization level solutions.}
\end{figure}
\begin{figure}[H]
\centering
\includegraphics[width=0.95\textwidth]{problem2/ctest_2d-2.png}
\caption{Output 2 of \code{ctest\_2d.m} - 
l-2 norm of exact error at each discretization level.}
\end{figure}

Using the scripts \code{video\_rec\_bar.m}, \code{video\_rec\_well.m}, and 
\code{video\_double\_slit.m}, videos of boosted Gaussian wave functions interacting with 
various produced. Please refer to Appendices G, H, and I for the full source code for these scripts,
including all simulation parameters.

In Figures 8 and 9 below, the interaction of the boosted Gaussian wave function on a rectangular
potential barrier is shown. The barrier is at a potential of 1e8 and is located in the black rectangle.
Figure 8, depicts the initial conditions, and Figure 9 depicts the 
reflection and scattering off of the well. This reflection is what we expect to see, which
indicates that the implementation of \code{sch\_2d\_adi.m} is correct. The full video is saved
as \code{rec\_bar.mp4}.

\begin{figure}[H]
\centering
\includegraphics[width=0.75\textwidth]{problem2/rec_bar_1.png}
\caption{Screenshot of video produced by \code{video\_rec\_bar.m} - Initial conditions.}
\end{figure}
\begin{figure}[H]
\centering
\includegraphics[width=0.75\textwidth]{problem2/rec_bar_2.png}
\caption{Screenshot of video produced by \code{video\_rec\_bar.m} - Scattering off barrier.}
\end{figure}

In Figures 10 and 11 below, the interaction of the boosted Gaussian wave function on a rectangular
potential well is shown. The well is at a potential of -1e8 and is located in the black rectangle.
Figure 10, depicts the initial conditions, and Figure 11 depicts the 
interaction with the well. We can see that the wave function reflects and scatters off of the well 
almost identically to that of the barrier. This is strange, as we expect the wave function to become
trapped inside the well as it is at a lower potential. This could indicate a sign error somewhere in 
\code{sch\_2d\_adi.m}, but this uncertain. The full video is saved as \code{rec\_well.mp4}.

\begin{figure}[H]
\centering
\includegraphics[width=0.75\textwidth]{problem2/rec_well_1.png}
\caption{Screenshot of video produced by \code{video\_rec\_well.m} - Initial conditions.}
\end{figure}
\begin{figure}[H]
\centering
\includegraphics[width=0.75\textwidth]{problem2/rec_well_2.png}
\caption{Screenshot of video produced by \code{video\_rec\_well.m} - Scattering off well.}
\end{figure}

In Figures 12 and 13 below, the interaction of the boosted Gaussian wave function on a thin double
slit potential is shown. The double slit potential is two $y$ indices thick, is symmetric about $x$, 
and has a potential of 1e8 where the black lines are drawn in Figures 12 and 13. In Figure 12, we 
see the initial conditions - a boosted Gaussian wave function centered at $y=0$ moving in the 
positive $y$ direction. In Figure 13, the effect of the double slit on this potential is shown - we 
can see that most of the wave function reflects back off the potential but some gets through. The 
part of the wave function that does get through undergoes self-interference, which is highlighted by 
the red rectangle in Figure 13. This is clearly seen as the magnitude of $|\psi|$ oscillating across
the $x$ axis. This gives some more confidence that the simulation is correct. For more information 
on this phenomenon please visit 
\href{https://en.wikipedia.org/wiki/Double-slit_experiment}{this Wikipedia page}. The full video is 
saved as \code{double\_slit.mp4}.

\begin{figure}[H]
\centering
\includegraphics[width=0.75\textwidth]{problem2/double_slit_1.png}
\caption{Screenshot of video produced by \code{double\_slit\_well.m} - Initial conditions.}
\end{figure}
\begin{figure}[H]
\centering
\includegraphics[width=0.75\textwidth]{problem2/double_slit_2.png}
\caption{Screenshot of video produced by \code{double\_slit\_well.m} - 
Self interference through slits.}
\end{figure}


%%%%%%%%%%%%%%%%%%%%%%%%%%%%%%%%%%%%%%%%%%%%%%%%%%%%%%%%%%%%%%%%%%%%%%%%%%%%%%%%%%%%%%%%%%%%%%%%%%%%%%%
% Conclusions
%%%%%%%%%%%%%%%%%%%%%%%%%%%%%%%%%%%%%%%%%%%%%%%%%%%%%%%%%%%%%%%%%%%%%%%%%%%%%%%%%%%%%%%%%%%%%%%%%%%%%%%
\subsection*{Conclusions}

In this project, the time-dependent Schrödinger equation in one and two dimensions was numerically 
computed, and simulations of 1d and 2d wave functions were produced. Convergence testing was done 
in both 1d and 2d, and additional numerical experiments were also conducted for both 1d and 2d to 
provide more insights on the accuracy of the produced numerical models and the underlying physical 
phenomena. 

Most convergence tests and numerical experiments yielded the expected results for 1d and 
2d, indicating the models are correct. However, in the 2d case the exact solution error was quite large
and did not show convergence among different discretization levels. Additionally, the simulation of a 
boosted Gaussian wave function incident on a rectangular well in 2d did not show the wave function 
becoming trapped in the well, but instead showed the well acting as a potential barrier. These two 
experiments indicate there may be an issue with the 2d numerical model.

Another issue present throughout this project is that of computation time. The barrier and well surveys
in 1d both took greater than 30 minutes to be computed, and the 2d simulations above $l=8$ took similar 
periods of time to compute. This issue could likely be mitigated by optimizing the simulation and test
code, for example by pre-allocating memory or utilizing parallel computation tools. This would allow 
for faster code development and debugging. 

Generative AI was used to help with understanding how to use MATLAB's \code{contourf} function for 
making videos of numerical experiments in 2d. It was also used for help with typesetting this 
document.

\pagebreak

%%%%%%%%%%%%%%%%%%%%%%%%%%%%%%%%%%%%%%%%%%%%%%%%%%%%%%%%%%%%%%%%%%%%%%%%%%%%%%%%%%%%%%%%%%%%%%%%%%%%%%%
% Appendices
%%%%%%%%%%%%%%%%%%%%%%%%%%%%%%%%%%%%%%%%%%%%%%%%%%%%%%%%%%%%%%%%%%%%%%%%%%%%%%%%%%%%%%%%%%%%%%%%%%%%%%%

\subsection*{Appendix A - sch\_1d\_cn.m Code}
\lstinputlisting{../src/problem1/sch_1d_cn.m}
\pagebreak

\subsection*{Appendix B - ctest\_1d.m Code}
\lstinputlisting{../src/problem1/ctest_1d.m}
\pagebreak

\subsection*{Appendix C - barrier\_survey.m Code}
\lstinputlisting{../src/problem1/barrier_survey.m}
\pagebreak

\subsection*{Appendix D - well\_survey.m Code}
\lstinputlisting{../src/problem1/well_survey.m}
\pagebreak

\subsection*{Appendix E - sch\_2d\_adi.m Code}
\lstinputlisting{../src/problem2/sch_2d_adi.m}
\pagebreak

\subsection*{Appendix F - ctest\_2d.m Code}
\lstinputlisting{../src/problem2/ctest_2d.m}
\pagebreak

\subsection*{Appendix G - video\_rec\_bar.m Code}
\lstinputlisting{../src/problem2/video_rec_bar.m}
\pagebreak

\subsection*{Appendix H - video\_rec\_well.m Code}
\lstinputlisting{../src/problem2/video_rec_well.m}
\pagebreak

\subsection*{Appendix I - video\_double\_slit.m Code}
\lstinputlisting{../src/problem2/video_double_slit.m}
\pagebreak

\end{document}